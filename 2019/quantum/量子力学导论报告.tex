\documentclass{article}

%%%%%%%%%%%%%%%%%%%%%%%%%%%%%%%%%%%%%%%%%
% Lachaise Assignment
% Structure Specification File
% Version 1.0 (26/6/2018)
%
% This template originates from:
% http://www.LaTeXTemplates.com
%
% Authors:
% Marion Lachaise & François Févotte
% Vel (vel@LaTeXTemplates.com)
%
% License:
% CC BY-NC-SA 3.0 (http://creativecommons.org/licenses/by-nc-sa/3.0/)
% 
%%%%%%%%%%%%%%%%%%%%%%%%%%%%%%%%%%%%%%%%%

%----------------------------------------------------------------------------------------
%	PACKAGES AND OTHER DOCUMENT CONFIGURATIONS
%----------------------------------------------------------------------------------------

\usepackage{amsmath,amsfonts,stmaryrd,amssymb} % Math packages

\usepackage{enumerate} % Custom item numbers for enumerations

\usepackage{url}

\usepackage[ruled]{algorithm2e} % Algorithms

\usepackage[framemethod=tikz]{mdframed} % Allows defining custom boxed/framed environments

\usepackage{listings} % File listings, with syntax highlighting
\lstset{
	basicstyle=\ttfamily, % Typeset listings in monospace font
}

%----------------------------------------------------------------------------------------
%	DOCUMENT MARGINS
%----------------------------------------------------------------------------------------

\usepackage{geometry} % Required for adjusting page dimensions and margins

\geometry{
	paper=a4paper, % Paper size, change to letterpaper for US letter size
	top=1.5cm, % Top margin
	bottom=2.5cm, % Bottom margin
	left=2.4cm, % Left margin
	right=2.4cm, % Right margin
	headheight=10pt, % Header height
	footskip=1.5cm, % Space from the bottom margin to the baseline of the footer
	headsep=1cm, % Space from the top margin to the baseline of the header
	% showframe, % Uncomment to show how the type block is set on the page
}

%----------------------------------------------------------------------------------------
%	FONTS
%----------------------------------------------------------------------------------------

\usepackage[utf8]{inputenc} % Required for inputting international characters
\usepackage[T1]{fontenc} % Output font encoding for international characters

% \usepackage{XCharter} % Use the XCharter fonts\usepackage{XCharter} % Use the XCharter fonts

%----------------------------------------------------------------------------------------
%	COMMAND LINE ENVIRONMENT
%----------------------------------------------------------------------------------------

% Usage:
% \begin{commandline}
%	\begin{verbatim}
%		$ ls
%		
%		Applications	Desktop	...
%	\end{verbatim}
% \end{commandline}

\mdfdefinestyle{commandline}{
	leftmargin=10pt,
	rightmargin=10pt,
	innerleftmargin=15pt,
	middlelinecolor=black!50!white,
	middlelinewidth=2pt,
	frametitlerule=false,
	backgroundcolor=black!5!white,
	frametitle={Command Line},
	frametitlefont={\normalfont\sffamily\color{white}\hspace{-1em}},
	frametitlebackgroundcolor=black!50!white,
	nobreak,
}

% Define a custom environment for command-line snapshots
\newenvironment{commandline}{
	\medskip
	\begin{mdframed}[style=commandline]
}{
	\end{mdframed}
	\medskip
}

%----------------------------------------------------------------------------------------
%	FILE CONTENTS ENVIRONMENT
%----------------------------------------------------------------------------------------

% Usage:
% \begin{file}[optional filename, defaults to "File"]
%	File contents, for example, with a listings environment
% \end{file}

\mdfdefinestyle{file}{
	innertopmargin=1.6\baselineskip,
	innerbottommargin=0.8\baselineskip,
	topline=false, bottomline=false,
	leftline=false, rightline=false,
	leftmargin=1cm,
	rightmargin=1cm,
	singleextra={%
		\draw[fill=black!10!white](P)++(0,-1.2em)rectangle(P-|O);
		\node[anchor=north west]
		at(P-|O){\ttfamily\mdfilename};
		%
		\def\l{3em}
		\draw(O-|P)++(-\l,0)--++(\l,\l)--(P)--(P-|O)--(O)--cycle;
		\draw(O-|P)++(-\l,0)--++(0,\l)--++(\l,0);
	},
	nobreak,
}

% Define a custom environment for file contents
\newenvironment{file}[1][File]{ % Set the default filename to "File"
	\medskip
	\newcommand{\mdfilename}{#1}
	\begin{mdframed}[style=file]
}{
	\end{mdframed}
	\medskip
}

%----------------------------------------------------------------------------------------
%	NUMBERED QUESTIONS ENVIRONMENT
%----------------------------------------------------------------------------------------

% Usage:
% \begin{question}[optional title]
%	Question contents
% \end{question}

\mdfdefinestyle{question}{
	innertopmargin=1.2\baselineskip,
	innerbottommargin=0.8\baselineskip,
	roundcorner=5pt,
	nobreak,
	singleextra={%
		\draw(P-|O)node[xshift=1em,anchor=west,fill=white,draw,rounded corners=5pt]{%
		Question \theQuestion\questionTitle};
	},
}

\newcounter{Question} % Stores the current question number that gets iterated with each new question

% Define a custom environment for numbered questions
\newenvironment{question}[1][\unskip]{
	\bigskip
	\stepcounter{Question}
	\newcommand{\questionTitle}{~#1}
	\begin{mdframed}[style=question]
}{
	\end{mdframed}
	\medskip
}

%----------------------------------------------------------------------------------------
%	WARNING TEXT ENVIRONMENT
%----------------------------------------------------------------------------------------

% Usage:
% \begin{warn}[optional title, defaults to "Warning:"]
%	Contents
% \end{warn}

\mdfdefinestyle{warning}{
	topline=false, bottomline=false,
	leftline=false, rightline=false,
	nobreak,
	singleextra={%
		\draw(P-|O)++(-0.5em,0)node(tmp1){};
		\draw(P-|O)++(0.5em,0)node(tmp2){};
		\fill[black,rotate around={45:(P-|O)}](tmp1)rectangle(tmp2);
		\node at(P-|O){\color{white}\scriptsize\bf !};
		\draw[very thick](P-|O)++(0,-1em)--(O);%--(O-|P);
	}
}

% Define a custom environment for warning text
\newenvironment{warn}[1][Warning:]{ % Set the default warning to "Warning:"
	\medskip
	\begin{mdframed}[style=warning]
		\noindent{\textbf{#1}}
}{
	\end{mdframed}
}

%----------------------------------------------------------------------------------------
%	INFORMATION ENVIRONMENT
%----------------------------------------------------------------------------------------

% Usage:
% \begin{info}[optional title, defaults to "Info:"]
% 	contents
% 	\end{info}

\mdfdefinestyle{info}{%
	topline=false, bottomline=false,
	leftline=false, rightline=false,
	nobreak,
	singleextra={%
		\fill[black](P-|O)circle[radius=0.4em];
		\node at(P-|O){\color{white}\scriptsize\bf i};
		\draw[very thick](P-|O)++(0,-0.8em)--(O);%--(O-|P);
	}
}

% Define a custom environment for information
\newenvironment{info}[1][Info:]{ % Set the default title to "Info:"
	\medskip
	\begin{mdframed}[style=info]
		\noindent{\textbf{#1}}
}{
	\end{mdframed}
}


\usepackage{fontspec}
\usepackage{multicol}
\usepackage{enumitem}
\usepackage{url}

\XeTeXlinebreaklocale "zh"
\XeTeXlinebreakskip = 0pt plus 1pt
\setmainfont{Noto Sans CJK SC}

\title{量子力学导论课程报告}

\author{张义飞}

\date{201821080630}

\begin{document}

\maketitle

\section{David P. DiVincenzo简介}
David P. DiVincenzo是美国理论物理学家。他是Jülich的PeterGrünberg研究所理论纳米电子学研究所所长,亚琛工业大学量子信息研究所教授。在Daniel Loss的帮助下,他于1997年提出了Loss-DiVincenzo量子计算机,它将量子点中的电子自旋用作量子比特。1996年,在他在IBM的研究期间,他发表了一篇论文“Topics in Quantum Computing”,其中概述了他预测的创建量子计算机所需的5个最低要求。它后来被称为“DiVincenzo Criteria”并影响了许多开发工作量子计算机的实验研究。

\section{论文翻译}
\subsection{多量子体中的Schrieffer-wolff变换}

Schrieffer-Wolff(SW)方法是退化扰动理论的一种形式,其中低能量有效哈密顿量$H_{eff}$通过解耦低能量和高能量子空间的单一变换从精确哈密顿量获得。我们在严密的结果上给出了一个独立的SW方法概述。我们首先从线性子空间之间的直接旋转的SW转换的精确定义开始论证。然后我们获得了$H_{eff}$的几个重要性质的基本证明,例如链式聚类定理。然后,我们研究从直接旋转的泰勒级数表示获得的SW变换的微扰形式。我们的微扰方法提供了用于计算$H_{eff}$的高阶修正的系统图技术。然后,我们将SW方法专门用于具有短程相互作用的量子自旋晶格。我们在文献中研究的两种不同版本的SW方法中获得了有效的低能哈密顿量之间的单一等效性。最后,我们推导出精确度的上界,n阶有效哈密顿量的基态能量近似于精确的基态能量。

\section{量子计算机领域现状简介}
随着计算机科学的发展,史蒂芬·威斯纳在1969年最早提出“基于量子力学的计算设备”。而关于“基于量子力学的信息处理”的最早文章则是由亚历山大·豪勒夫(1973)、帕帕拉维斯基(1975)、罗马·印戈登(1976)和尤里·马尼(1980)年发表。史蒂芬·威斯纳的文章发表于1983年。1980年代一系列的研究使得量子计算机的理论变得丰富起来。1982年,理查德·费曼在一个著名的演讲中提出利用量子体系实现通用计算的想法。紧接着1985年大卫·杜斯提出了量子图灵机模型。人们研究量子计算机最初很重要的一个出发点是探索通用计算机的计算极限。当使用计算机模拟量子现象时,因为庞大的希尔伯特空间而数据量也变得庞大。一个完好的模拟所需的运算时间则变得相当长,甚至是不切实际的天文数字。理查德·费曼当时就想到如果用量子系统所构成的计算机来模拟量子现象则运算时间可大幅度减少,从而量子计算机的概念诞生。半导体靠控制集成电路来记录及运算信息,量子计算机则希望控制原子或小分子的状态,记录和运算信息。

量子计算机在1980年代多处于理论推导状态。1994年彼得·秀尔提出量子质因数分解算法后,证明量子计算机能做出离散对数运算,而且速度远胜传统计算机。因为量子不像半导体只能记录0与1,可以同时表示多种状态。如果把半导体比喻成单一乐器,量子计算机就像交响乐团,一次运算可以处理多种不同状况,因此,一个40比特的量子计算机,就能在很短时间内解开1024位计算机花上数十年解决的问题。因其对于现在通行于银行及网络等处的RSA加密算法可以破解而构成威胁之后,量子计算机变成了热门的话题,除了理论之外,也有不少学者着力于利用各种量子系统来实现量子计算机。

一般认为量子计算机仍处于研究阶段。然而2011年5月11日加拿大的D-Wave 系统公司发布了一款号称“全球第一款商用型量子计算机”的计算设备“D-Wave One”,含有128个量子位。2011年5月25日,洛克希德·马丁同意购买D-Wave One。南加州大学洛克希德马丁量子计算机研究中心(USC-Lockheed Martin Quantum Computation Center)证明D-Wave One不遵循古典物理学法则的模拟退火(simulated annealing)运算模型,而是量子退火法。该论文《可编程量子退火的实验特性》(Experimental Signature of Programmable Quantum Annealing)发表于《自然通信》(Nature Communications)期刊。该量子设备是否真的实现了量子计算当前还没有得到学术界广泛认同,只能有证据显示D-Wave系统在运作时逻辑不同于传统计算机。

2013年5月D-Wave 系统公司宣称NASA和Google共同预定了一台采用512量子位的D-Wave Two量子计算机。该计算机运行特定算法时比传统计算机快上亿倍,但换用算法解相同问题时却又输给传统计算机,所以实验色彩浓厚并延续了学术界争论。

2013年5月,谷歌和NASA在加利福尼亚的量子人工智能实验室发布D-Wave Two。

2015年5月,IBM在量子运算上获取两项关键性突破,开发出四量子比特型电路(four quantum bit circuit),成为未来10年量子计算机基础。另外一项是,可以同时发现两项量子的错误类型,分别为bit-flip(比特翻转)与phase-flip(相位翻转),不同于过往在同一时间内只能找出一种错误类型,使量子计算机运作更为稳定。

2015年10月,新南威尔士大学首度使用硅制作出量子闸。

2016年8月,美国马里兰大学学院市分校发明世界上第一台由5量子比特组成的可编程量子计算机。

2017年5月,中国科学院宣布制造出世界首台超越早期经典计算机的光量子计算机,研发了10比特超导量子线路样品,通过高精度脉冲控制和全局纠缠操作,成功实现了当前世界上最大数目的超导量子比特多体纯纠缠,并通过层析测量方法完整地刻画了十比特量子态。[20]此原型机的“玻色取样”速度比国际同行之前所有实验机加快至少24000倍,比人类历史上第一台电子管计算机(ENIAC)和第一台晶体管计算机(TRADIC)运行速度快10-100倍,虽然还是缓慢但已经逐步跨入实用价值阶段。

2017年7月,美国研究人员宣布完成51个量子比特的量子计算机模拟器。哈佛大学米哈伊尔·卢金(Mikhail Lukin)在莫斯科量子技术国际会议上宣布这一消息。量子模拟器使用了激光冷却的原子,并使用激光将原子固定。

2018年6月,英特尔宣布开发出新款量子芯片,使用五十奈米的量子比特做运算,并已在摄氏零下273度的极低温度中进行测试。

2019年1月8日,IBM在消费电子展(CES)上展示了已开发的世界首款商业化量子计算机IBM Q System One。

2019年3月12日,IBM发布了量子性能的“摩尔定律”。

\section{量子计算机领域五位科学家}
D Deutsch, R Raussendorf, BE Kane, PW Shor, MA Nielsen

\section{量子计算与量子信息科学领域的个人看法}
我认为量子计算技术是下一代计算平台发展的方向,当前的通用计算机对于NP问题的求解还处于非常原始的阶段,而量子计算机由于量子本身的特性,可以在非常好的时间复杂度内解决这些问题。然而受制作工艺的限制,当前量子计算机的计算性能还远没有达到可用水平,尽管如此,量子计算领域依然是一个非常有发展前景的领域。

\section{《量子计算与量子信息科学导论》课程的建议}
这门课程虽说是导论性质的课,但其涉及的量子力学理论部分对于计算机方向或其他方向的学生还是有一定的理解门槛。目前微软提出了一种量子力学的通用编程语言Q\#,我认为下次开设此课程可以引入Q\#的介绍以及简单使用,该语言目前可以在通用计算机上模拟量子计算机的某些行为,所以可以帮助同学们理解量子计算机的运算机制。

\end{document}
