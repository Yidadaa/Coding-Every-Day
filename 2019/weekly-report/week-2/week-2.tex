\documentclass{article}

    \input{../template/structure.tex}
    
    \title{CFM AI\&VR Weekly Report: Week \#2 of 2019}
    
    \author{Zhang Yifei(\texttt{yidadaa@qq.com})}
    
    \date{UESTC --- \today}
    
    \begin{document}
    
    \maketitle

    \section{Summary}
    Last sunday, I finished the last examination of this semester. And then I immediately started the survey of SLAM. 

    \section{Paper Reading: CNN-SLAM}
    \begin{figure}[h]
        \centering
        \includegraphics[width=0.6\textwidth,natwidth=1273,natheight=365]{./img/cnn-slam-overview.aux}
    \end{figure}

    CNN-SLAM is a direct monolular SLAM algorithm, which is inspired by LSD-SLAM.
    \subsection{Main Methods}
    \begin{itemize}
        \item The camera pose estimation is inspired by the key-frame approach in LSD-SLAM. But in contrast to LSD-SLAM, CNN-SLAM uses a CNN-based depth prediction to generate the depth map, which could validly solve the limitations of traditional monolular SLAM.
        \item The main contribution of CNN-SLAM is the scheme employed to refine the CNN-predicted depth map associated to each key-frame via small-baseline stero matching.
        \item The framework of CNN-SLAM is capable of jointly reonstructing the scene while fusing semantic segmentation labels.
    \end{itemize}

    \subsection{Conclusion}
    \begin{itemize}
        \item CNN-SLAM can run in real-time since the two processes of depth prediction from CNNs run on GPU and other stages run on CPU.
        \item CNN-SLAM overcomes many limitations of traditional SLAM, especially with respect to estimating the absolute scale, obtaining dense depths along texture-less regions andg dealing with pure rotational motions.
    \end{itemize}
    \section{Plan}
    I will continue reading more papers and textbooks about SLAM. Here is a list of to-be-read materiels:
    \begin{itemize}
        \item Paper: LSD-SLAM: Large-Scale Direct Monocular SLAM
        \item Paper: ORB-SLAM2: an Open-Source SLAM System for Monocular, Stereo and RGB-D Cameras
        \item Textbook: <14 lectures on visual SLAM>
    \end{itemize}
    \end{document}
