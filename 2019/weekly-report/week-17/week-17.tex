\documentclass{article}

\input{../template/structure.tex}

\usepackage{url}

\title{CFM AI\&VR Weekly Report: Week \#17 of 2019}

\author{Zhang Yifei(\texttt{yidadaa@qq.com})}

\date{UESTC --- \today}

\begin{document}

\maketitle

\section{Summary}
This week I took a deep insight into Intel RealSense platform and Nvidia Jetson platform, and tried to figure out the possibility of deploying deep learning algorithm on them.

\section{Intel Realsense D400-Series Depth Camera}
The Intel Realsense Depth Camera D400-series uses stereo vision to calculate depth. It's easy to get RGB image with depth using the camera. Intel developers provide a very convient Python binding of the API. \textbf{Resources}: \url{https://github.com/IntelRealSense/librealsense/tree/master/wrappers/python}.
\subsection{Application}
Self-Driving Bot, Hand Gesture Recognition, Edge-Based Intelligence

\section{Nvidia Jetson Series Platform}
Nvidia Jetson Series Platform integrates hundreds of CUDA units on embedded system, so that we can build AI algorithms at the edge. Nvidia Jetson series platform can provide one-sixth of the performance of Nvidia GTX 1080Ti. Nvidia prepares lots of tutorials and SDK to help developers use the platform.

\subsection{Application}
Edge-Based Intelligence, AIoT

\begin{figure}[h]
    \centering
    \includegraphics[width=0.8\textwidth]{./img.png}
    \caption{Different type of Jetson series product.}
\end{figure}

\section{Plan}
\begin{itemize}
    \item Start developing on Intel Realsense and Nvidia Jetson platforms if possible.
\end{itemize}

\end{document}
