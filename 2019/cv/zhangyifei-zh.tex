\documentclass[10pt]{article}

\usepackage{ifthen}
\usepackage{geometry}
\usepackage[usenames,dvipsnames]{color}
\usepackage{marvosym}
\usepackage{wasysym}
\usepackage{enumitem}
\usepackage{setspace}
\usepackage{url}
\usepackage{multicol}
\usepackage{fontspec}
\usepackage{xeCJK}
\usepackage{changepage}
\usepackage{listings}

\thispagestyle{empty}\pagestyle{empty}

\XeTeXlinebreaklocale "zh"
\setCJKmainfont{Source Han Serif SC}

\input{cv.tex}

\begin{document}

\cvtitle{\textbf{张义飞}}{BrickRed}{前端 / 计算机视觉}{0.5cm}

\positionedbox{left}{0.7\textwidth}{%
	电话:+86 13032883129\\
	邮箱:flynn.zhang@foxmail.com\\
	博客:blog.simplenaive.cn\\
	Github:github.com/Yidadaa
}
\positionedbox{right}{0.3\textwidth}{DOB: 20/01/97\\Chengdu, China}

\vspace{1cm}

\begin{multicols}{2}
		\cvsectiontitle{\textbf{教育背景}}
		\vspace{-1cm}
		\begin{itemize}[leftmargin=0.5cm]
			\item \cvcompany{硕士,电子科技大学}{2018.09 - 2021.06}\\\vspace{-0.3cm}\\
						计算机科学与工程学院,计算机科学专业
			\vspace{0.3cm}
			\item \cvcompany{学士,电子科技大学}{2014.09 - 2018.06}\\\vspace{-0.3cm}\\
						英才实验学院(工科试验班),计算机科学专业
		\end{itemize}
	
		\vspace{0.4cm}
		\cvsectiontitle{\textbf{实习\&研究经历}}
		\vspace{-1cm}
		\begin{itemize}[leftmargin=0.5cm]
		\item \cvcompany{三维视觉中的深度估计算法研究}{from 2019.04}\cvsublevel{%
			\cvsubbullet{%
				研究方向是基于单目视频序列的无监督深度估计算法,原因是基于深度神经网络构建的%
				无监督训练流程可以很快地被迁移到全新环境中,并且训练时只需要单目摄像头视频数据,非常有应用前景。%
			}%
			\cvsubbullet{%
				致力于从计算机视觉中其他任务(如无监督视频语义分割)的方法中汲取灵感,来提升深度估计训练流程的性能,通过调研顶会%
				论文,已经对无监督深度估计流程有了全面认识,目前在以实验的方式验证 co-attention 模块对现有%
				无监督深度估计流程带来的性能影响。
			}
			\cvsubbullet{%
				\textbf{关键字:机器学习和深度学习基础,深度学习项目实践}
			}
		}
		\item \cvcompany{前端开发工程师 @ 上海尘微科技}{2017.12 - 2019.3}\cvsublevel{%
			\cvsubbullet{远程开发,兼职参与}
			\cvsubbullet{%
				该小程序是一个包含数十个页面及模块的中型工程项目,完全托管于 Github 的私有库,所有成员均为远程开发%
				者,通过 Github 看板功能和 Issue 来追踪 Bugs \& Features 的开发进展。
			}%
			\cvsubbullet{%
				本人参与负责了两处重要的面向用户的首屏和社区页面的开发工作,协调设计师、%
				产品经理以及前后端开发人员进行需求评估、编码、对接、上线交付以及 Bug 追踪等工作,%
				主要技术难点包括首屏动态布局、Feed 流长列表优化等。
			}
			\cvsubbullet{%
				\textbf{关键字:团队协调、中型项目开发及管理经验、Git 工作流、小程序开发技术、Vue}
			}
		}
		\item \cvcompany{前端开发实习生 @ 百度}{2016.12 - 2017.5}\cvsublevel{%
			\cvsubbullet{%
				百度凤巢部门负责开发面向广告主的广告管理后台系统,该系统是一个包含数百个页面和模块以及各种复杂交互逻辑的大型工程项目,%
				凤巢前端部门每周与后端、测试、产品经理等部门协调需求。
			}%
			\cvsubbullet{%
				本人在实习期间负责了广告智能推荐前端界面的开发,并参与了该需求的正式上线和后续维护工作,工作期间%
				对 Bug 响应迅速,产出的代码可维护性强且十分健壮,受到 mentor 的一致好评。
			}%
			\cvsubbullet{%
				\textbf{关键字:中型团队合作、大型项目开发经验、成熟团队工作流程、React \& Redux}
			}
		}
		\end{itemize}

		\vspace{0.4cm}
		\cvsectiontitle{\textbf{开源项目\&编程能力}}
		\vspace{-0.8cm}

		\cvcompany{4 Stars}{2019.04}\\%
			\rightline{\textit{\small{github.com/Yidadaa/Pytorch-Video-Classification}}}
			\rightline{\textit{\small{\textbf{Python / Pytorch}, $\approx 500$ lines}}}
			基于 CNN-RNN 架构的视频分类网络,在 UCF101 上达到 80\% 的准确率。

		\cvcompany{}{2019.05}\\%
			\rightline{\textit{\small{github.com/Yidadaa/Satellite-Imagery-Segmantation-Deeplab}}}
			\rightline{\textit{\small{\textbf{Python / Pytorch}, $\approx 1000$ lines}}}
			阿里天池比赛代码,使用 Deeplabv3 对超大卫星图做分割。

		\cvcompany{}{2018.11}\\
			\rightline{\textit{\small{github.com/Yidadaa/Parallel-Programming-On-GPU}}}
			\rightline{\textit{\small{\textbf{CUDA / C++}, $\approx 200$ lines}}}
			使用 CUDA 加速 n-body 模拟程序,成功加速到原来的 3000 多倍。

		\cvcompany{3 Stars}{2018.01}\\%
			\rightline{\textit{\small{github.com/Yidadaa/Captcha-Deep-Learning}}}
			\rightline{\textit{\small{\textbf{Python / Keras / Tensorflow}, $\approx 500$ lines}}}
			搭建端到端的验证码识别网络,分别使用 Keras 和 Tensorflow 实现,并对比差异,准确率达到 98\%。

		\cvcompany{}{2019.03}\\%
			\rightline{\textit{\small{github.com/Yidadaa/OPPO-Human-Segmentation}}}
			\rightline{\textit{\small{\textbf{C++}, $\approx 1000$ lines}}}
			OPPO AI 挑战赛 Demo 源码,将人像语义分割网络经过腾讯开源的 ncnn 框架转换后部署到移动端。

		\cvcompany{4 Stars}{2018.11}\\%
			\rightline{\textit{\small{github.com/Yidadaa/P2P-Message}}}
			\rightline{\textit{\small{\textbf{Dart / Python}, $\approx 1000$ lines}}}
			P2P 聊天软件,使用 Dart 语言构建客户端,使用 Python 构建服务端。

		\cvcompany{1 Stars}{2019.05}\\%
			\rightline{\textit{\small{github.com/Yidadaa/HUAWEI-Codecraft-2019}}}
			\rightline{\textit{\small{\textbf{C++}, $\approx 1000$ lines}}}
			华为软件精英挑战赛代码,车辆路径智能规划,包含完整的单元测试代码,完全遵循 Google C++ Style Guide 规范构建。
		
		\vspace{0.7cm}
		\cvsectiontitle{\textbf{荣誉\&奖项}}
		\vspace{-0.8cm}

		\cvcompany{\small{研究生二等学业奖学金}}{2019.10}\\%
		\cvcompany{\small{OPPO AI 挑战赛人像分割任务\ 决赛优秀奖}}{2019.04}\\%
		\cvcompany{\small{OPPO AI 挑战赛人像分割任务\ 复赛第一名}}{2019.03}\\%
		\cvcompany{\small{研究生一等学业奖学金}}{2018.10}\\%
		\cvcompany{\small{美国大学生数学建模竞赛 H 奖}}{2017.02}\\%

		\vspace{1cm}\rightline{*}

\end{multicols}

\end{document}
