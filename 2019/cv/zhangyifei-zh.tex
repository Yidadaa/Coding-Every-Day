\documentclass[10pt]{article}

\usepackage{ifthen}
\usepackage{geometry}
\usepackage[usenames,dvipsnames]{color}
\usepackage{marvosym}
\usepackage{wasysym}
\usepackage{enumitem}
\usepackage{setspace}
\usepackage{url}
\usepackage{multicol}
\usepackage{paracol}
\usepackage{fontspec}
\usepackage{xeCJK}
\usepackage{changepage}
\usepackage{listings}

\thispagestyle{empty}\pagestyle{empty}

\XeTeXlinebreaklocale "zh"
\setCJKmainfont{Source Han Serif SC}

\input{cv.tex}

\begin{document}

\cvtitle{\textbf{张义飞}}{BrickRed}{三维视觉}{0.5cm}

\positionedbox{left}{0.7\textwidth}{%
	电话:+86 13032883129\\
	邮箱:\url{flynn.zhang@foxmail.com}\\
	博客:\url{https://blog.simplenaive.cn}\\
	Github:\url{https://github.com/Yidadaa}
}
\positionedbox{right}{0.3\textwidth}{DOB: 20/01/97\\Chengdu, China}

\vspace{0.7cm}

\setlength{\columnsep}{1cm}
\begin{multicols}{2}
		\cvsectiontitle{\textbf{教育背景}}
		\cvcompany{硕士,电子科技大学}{2018.09 - 2021.06}\\
		计算机科学与工程学院,计算机科学专业

		\cvcompany{学士,电子科技大学}{2014.09 - 2018.06}\\
		英才实验学院(工科试验班),计算机科学专业

		\columnbreak

		\cvsectiontitle{\textbf{荣誉\&奖项}}
		\vspace{-0.5cm}

		\cvcompany{\small{研究生二等学业奖学金}}{2019.10}\\%
		\cvcompany{\small{OPPO AI 挑战赛人像分割任务\ 决赛优秀奖}}{2019.04}\\%
		\cvcompany{\small{OPPO AI 挑战赛人像分割任务\ 复赛第一名}}{2019.03}\\%
		\cvcompany{\small{研究生一等学业奖学金}}{2018.10}\\%
		\cvcompany{\small{美国大学生数学建模竞赛 H 奖}}{2017.02}
\end{multicols}
	
		\vspace{0.5cm}
		\cvsectiontitle{\textbf{实习\&研究经历}}
		\vspace{-1cm}
		\begin{itemize}[leftmargin=0.5cm]
		\item \cvcompany{三维重建以及点云传输算法 @ 腾讯 Robotics X}{from 2019.11}\cvsublevel{%
			\cvsubbullet{%
				主要负责调研三维重建算法,以及解决动态场景下三维重建算法的可用性。%
			}% 
			\cvsubbullet{%
				前期优化 Elastic Fusion 重建算法,使用 C++ 为其编写 RealSense 驱动,同时负责 VoxBlox 重建算法与 ORB-SLAM%
				的对接,使用 C++ 编写 ROS 接口,并调研各种数据传输算法的性能优劣。
			}
			\cvsubbullet{%
				随后基于 Elastic Fusion 重建算法,配合实时动态标定算法,实现 VR 环境下的机器臂遥操作,主要负责%
				开发从重建端到展示端的点云数据传输以及点云渲染算法,并对其实时性、丢包率等性能设计实验进行验证。
			}
		}
		% \item \cvcompany{前端开发工程师 @ 上海尘微科技}{2017.12 - 2019.3}\cvsublevel{%
		% 	\cvsubbullet{远程开发,兼职参与,负责了两处重要的面向用户的首屏和社区页面的开发工作,协调小型团队从需求分析到上线的全套流程,并负责攻克技术难点,%
		% 		主要技术难点包括首屏动态布局、Feed 流长列表优化等。
		% 	}
		% 	\cvsubbullet{%
		% 		\textbf{关键字:团队协调、中型项目开发经验、Git 工作流、小程序开发技术、Vue}
		% 	}
		% }
		\item \cvcompany{基于 Unity 虚拟环境和强化学习的机械臂控制算法 @ 毕业设计}{2017.10 - 2018.06}\cvsublevel{%
			\cvsubbullet{强化学习算法在机械智能控制中愈发重要,然而强化学习算法往往需要在训练阶段通过大量地 exploration 与环境交互取得数据来优化策略,这种训练策略将会在
				真实世界带来极高的时间和物料成本,本文提出了一种架构,可以使得现有强化学习模型可以实时与 Unity 中的虚拟环境交互并获取训练数据,为迁移至真实环境提供预训练基础。
			}
			\cvsubbullet{主要工作如下:1) 构建 Unity 运行时和 Python 运行时的中间层,使得使用 Tensorflow 和 Pytorch 的构建的
				深度学习模型可以与 Unity 虚拟环境交互;2) 实现了强化学习中经典的 PPO 算法,并分别在二维和三维环境中设计对应的任务来验证算法和中间层的可用性;
				3) 基于本文提出的架构,探讨了多线程以及离屏和低分辨率渲染等手段对强化学习训练任务的加速能力。
			}
		}
		\item \cvcompany{前端开发实习生 @ 百度}{2016.12 - 2017.05}\cvsublevel{%
			\cvsubbullet{%
				主要负责百度凤巢的广告智能推荐系统的开发,参与了相关需求的正式上线和后续维护的整套流程,工作期间通过与其他部门紧密合作,%
				对 Bug 迅速响应,并产出可维护强的、健壮的工程代码。
			}%
			\cvsubbullet{%
				\textbf{关键字:中型团队合作、大型项目开发经验、React \& Redux}
			}
		}
		\end{itemize}

		\vspace{0.4cm}
		\cvsectiontitle{\textbf{开源项目\&编程能力}}
		\vspace{-0.8cm}

		\begin{adjustwidth}{0.1cm}{0cm}
			\cvcompany{\project{github.com/Yidadaa/Pytorch-Video-Classification}{Python / Pytorch}{$\approx 500$ lines}}{2019.04}\\
				基于 CNN-RNN 架构的视频动作分类网络,在 UCF101 上达到 80\% 的准确率。

			\cvcompany{\project{github.com/Yidadaa/Satellite-Imagery-Segmantation-Deeplab}{Python / Pytorch}{$\approx 1000$ lines}}{2019.05}\\
				阿里天池比赛代码,使用 Deeplabv3 对超大卫星图做分割。

			\cvcompany{\project{github.com/Yidadaa/Parallel-Programming-On-GPU}{CUDA / C++}{$\approx 200$ lines}}{2018.11}\\
				使用 CUDA 加速 n-body 模拟程序,加速比达到 3000。

			\cvcompany{\project{github.com/Yidadaa/Captcha-Deep-Learning}{Python / Keras / Tensorflow}{$\approx 500$ lines}}{2018.01}\\
				搭建端到端的验证码识别网络,分别使用 Keras 和 Tensorflow 实现,在测试集上达到 98\% 的准确率。

			\cvcompany{\project{github.com/Yidadaa/OPPO-Human-Segmentation}{C++ / Dart}{$\approx 1000$ lines}}{2019.03}\\
				OPPO AI 挑战赛 Demo 源码,将人像语义分割网络经过腾讯开源的 ncnn 框架转换后部署到移动端。

			\cvcompany{\project{github.com/Yidadaa/HUAWEI-Codecraft-2019}{C++}{$\approx 1000$ lines}}{2019.05}\\
				华为软件精英挑战赛代码,车辆路径智能规划,包含完整的单元测试,严格遵循 Google C++ Style Guide 规范。
		\end{adjustwidth}

\end{document}
